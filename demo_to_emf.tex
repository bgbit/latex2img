% 这是一个将tikz图片转换成多张EMF的示例文件,使用inkscape来实现转换
% 這是一個將tikz圖片轉換成多張EMF的示例文件,使用inkscape來實現轉換
% This is a demo file for tikz to multiple EMFs using inkscape
\documentclass[tikz, convert, convert={outext=.pdf, command=\unexpanded{
% 'out_emf'是用来存放EMF的文件夹
% 'out_emf'是用來存放EMF的文件夾
% 'out_emf' is the destination folder for EMF files
call ./util/mk_folder out_emf
&& call ./util/gs_split_pdf.bat out_emf \outfile\space \infile\space
&& cd /d out_emf
&& call ../util/pdf_to_emf.bat
&& del /F *.pdf \sapce
}}]{standalone}
% inkscape只能实现单张的PDF转换EMF,所以要先用ghostscript把LaTex生成的
% PDF分页,然后调用inkscape做循环,把所有单页的PDF转换为EMF,最后删除所
% 有单页的PDF,只保留EMF。
% inkscape只能實現單張的PDF轉換EMF,所以要先用ghostscript把LaTex生成的
% PDF分頁,然後調用inkscape做循環,把所有單頁的PDF轉換爲EMF,最後刪除所
% 有單頁的PDF,只保留EMF。
% inkscape can only convert single page PDF to EMF. Therefore, the whole
% PDF generated by LaTex needs to be split into single pages first, by
% ghostscript. Then use inkscape in a loop to convert all single page
% PDFs into EMFs. Finally, delete all single page PDFs and keep only the
% EMFs.

\usepackage{xeCJK}
\setCJKmainfont{Microsoft YaHei}

\usepackage{scalefnt}
\usepackage{tikz}

% tikz and colour configs
% 以下是关于tikz中画流程图的设置
% 以下是關於tikz中畫流程圖的設置
% configure flowchart shapes
\usetikzlibrary{shapes.geometric, arrows, positioning, calc}

% start, end shape
\tikzstyle{startstop} = [rectangle, rounded corners, minimum width=3cm,
minimum height=1cm,text centered,  text=white, draw=black,
fill=colorStarstop]

% process shape
\tikzstyle{process} = [rectangle, minimum width=3cm, minimum height=1cm,
text centered, text=white, draw=black, fill=colorPro]

% decision shape
\tikzstyle{decision}=[diamond, minimum width=3cm, minimum height=1cm,
text centered, draw=black, fill=colorDec]

% comment shape
\tikzstyle{comment}=[dashed, draw=black, fill=gray!10, minimum width=3cm, minimum height=1cm, text centered]

% docstring shape
\tikzstyle{docstring}=[draw=orange, fill=white, minimum width=50mm, text width=80mm, minimum height=1cm]

% arrows shape
\tikzstyle{arrow} = [ultra thick,->,>=stealth, line width=1.5mm]

% comment shape
\tikzstyle{comment}=[dashed, draw=black, fill=gray!10, minimum width=3cm, minimum height=1cm, text centered]

\tikzset{
    subprocess/.style = {rectangle, draw=black, semithick, fill=orange!30,
        minimum width=#1, minimum height=1cm, inner xsep=3mm, % <-- changed
        text width =\pgfkeysvalueof{/pgf/minimum width}-2*\pgfkeysvalueof{/pgf/inner xsep},
        align=flush center,
        path picture={\draw
            ([xshift =2mm] \ppbb.north west) -- ([xshift= 2mm] \ppbb.south west)
            ([xshift=-2mm] \ppbb.north east) -- ([xshift=-2mm] \ppbb.south east);
        },
    },
    subprocess/.default = 24mm % <-- added
}% end of tikzset

\usetikzlibrary{positioning}

% 以下是颜色的设置
% 以下是顏色的設置
% colour defs
\usepackage{color}
\definecolor{colorStarstop}{RGB}{174, 23, 21}
\definecolor{colorPro}{RGB}{0, 175, 121}
\definecolor{colorDec}{RGB}{255, 192, 0}
\definecolor{colorYes}{RGB}{51, 153, 51}
\definecolor{colorNo}{RGB}{255, 0, 0}

\begin{document}

    % 全局字体缩放
    % 全局字體縮放
    % global font scale
    \scalefont{1.3}

    % tikz图像文档
    % tikz圖像文檔
    % tikz pics file
    \begin{tikzpicture}[baseline=(current bounding box.north), scale=1.0, every node/.style={scale=1}]

    \matrix[column sep=15mm, row sep=15mm]
    {
        % row 1
        & \node (start)[startstop]{开始/開始}; &  \\

        % row 2
        & \node (p1)[process, text width=40mm]{流程1/流程1};
        & \node (c1)[comment, text width=40mm]{注释1/註釋1};
        & \\

        % row 3
        & \node (p2)[process, text width=40mm]{流程2/流程2};
        & \node (c2)[comment, text width=40mm]{注释2/註釋2};
        & \\

        % row4
        & \node (p3)[process, text width=40mm]{流程3/流程3};
        & \node (c3)[comment, text width=40mm]{注释3/註釋3};
        & \\

        % row5
        & \node (dec)[decision, text width=20mm]{判断/\\判斷};
        & \\

        % row 6
        & \node (p4)[process, text width=40mm]{流程4/流程4};
        & \node (c4)[comment, text width=40mm]{注释4/註釋4};
        & \\

        % row 7
        & \node (end)[startstop]{结束/結束};
        & \\
    };

    % lines and arrows
    \draw[arrow](start) -- (p1);
    \draw[arrow](p1) -- (p2);
    \draw[arrow](p2) -- (p3);
    \draw[arrow](p3) -- (dec);

    \draw[arrow](p4) -- (end);

    % 拐弯用/拐彎用
    \coordinate[right of=p1, xshift=-50mm] (dummy1);
    \draw[arrow, color=colorNo](dec)node[anchor=south east, xshift=-20mm]{非/非} -| (dummy1) -- (p1);

    \draw[arrow, color=colorYes](dec)node[anchor=south west, xshift=-18mm, yshift=-25mm]{是/是} -- (p4);

    % dashed line
    \draw[dashed] (p1) -- (c1);
    \draw[dashed] (p2) -- (c2);
    \draw[dashed] (p3) -- (c3);
    \draw[dashed] (p4) -- (c4);

\end{tikzpicture}

% 第二张图
% 第二張圖
\begin{tikzpicture}[baseline=(current bounding box.north), scale=1.0, every node/.style={scale=1.0}]

    \matrix[column sep=15mm, row sep=15mm]
    {
        % row 1
        & \node (start)[startstop]{开始/開始}; &  \\

        % row 2
        & \node (p1)[process, text width=40mm]{流程1/流程1};
        & \node (c1)[comment, text width=40mm]{注释1/註釋1};
        & \\

        % row 3
        &
        & \node (p2)[process, text width=40mm]{流程2/流程2};
        & \\

        % row4
        & \node (c2)[comment, text width=40mm]{注释2/註釋2};
        & \node (c3)[comment, text width=40mm]{注释3/註釋3};
        & \node (p3)[process, text width=40mm]{流程3/流程3};
        & \\

        % row5
        &
        &
        & \node (dec)[decision, text width=20mm]{判断/\\判斷};
        & \\

        % row 6
        &
        & \node (c4)[comment, text width=40mm]{注释4/註釋4};
        & \node (p4)[process, text width=40mm]{流程4/流程4};
        & \\

        % row 7
        &
        &
        & \node (end)[startstop]{结束/結束};
        & \\};

    % lines and arrows
    \draw[arrow](start) -- (p1);
    \draw[arrow](p1) |- (p2);
    \draw[arrow](p2) -| (p3);
    \draw[arrow](p3) -- (dec);

    \draw[arrow](p4) -- (end);

    % 拐弯用/拐彎用
    \coordinate[right of=p1, xshift=-50mm] (dummy1);
    \draw[arrow, color=colorNo](dec)node[anchor=south east, xshift=-20mm]{非/非} -| (dummy1) -- (p1);

    \draw[arrow, color=colorYes](dec)node[anchor=south west, xshift=-18mm, yshift=-25mm]{是/是} -- (p4);

    % dashed line
    \draw[dashed] (p1) -- (c1);
    \draw[dashed] (p2) -- (c2);
    \draw[dashed] (p3) -- (c3);
    \draw[dashed] (p4) -- (c4);

\end{tikzpicture}

\end{document}
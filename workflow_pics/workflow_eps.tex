% 转换为EPS之流程\\ 轉換爲EPS之流程\\ Workflow for Converting to EPS
\begin{tikzpicture}[baseline=(current bounding box.north), scale=1.0, every node/.style={scale=1}]

    \matrix[column sep=10mm, row sep=10mm]
    {
        % row 1
        & \node
        (start)[startstop]{$\overset{\overset{\text{Start EPS Conversion}}{\text{開始
        轉換EPS}}}{\text{开始转换EPS}}$};
        &  \\

        % row 2
        & \node (p1)[process, text
        width=60mm]{$\overset{\overset{\text{Make folder "out\_eps" under
        the root directory}}{\text{在根目錄下創建“out\_eps”文件
        夾}}}{\text{在根目录下创建“out\_eps”文件夹}}$};
        & \node(c1)[comment]{$\overset{\overset{\text{Will not create if
        folder already exists}}{\text{若已存在,則不創建}}}{\text{若已存在,则不创建}}$};
        & \\

        % row 3
        & \node (dec1)[decision, text
        width=20mm]{$\overset{\overset{\text{Successful?}}{\text{成
        功?}}}{\text{成功?}}$};
        & \node (p2)[process, text
        width=140mm]{$\overset{\overset{\text{\scalefont{0.8}Call the "gs\_split\_pdf"
        script in the "util" directory to split the generated PFD into
        single-page PDFs and put them into the "out\_eps"
        folder}}{\text{\scalefont{0.8}調用“util”目錄中之“gs\_split\_pdf”腳本轉換成將生
        成的PDF分割成單頁並導出到“out\_eps”文件夾中}}}{\text{\scalefont{0.8}调用“util”
        目录中之“gs\_split\_pdf”脚本转换成将生成的PDF分割成单页并导出到
        “out\_eps”文件夹中}}$};
        & \\

        % row 4
        &
        & \node (dec2)[decision, text
        width=20mm]{$\overset{\overset{\text{Successful?}}{\text{成
        功?}}}{\text{成功?}}$};
        & \\

        % row 5
        &
        & \node (p3)[process, text
        width=40mm]{$\overset{\overset{\text{Enter the "out\_eps"
        directory}}{\text{進入“out\_eps”目錄}}}{\text{进入“out\_eps”目
        录}}$};
        & \\

        % row 6
        &
        & \node (dec3)[decision, text
        width=20mm]{$\overset{\overset{\text{Successful?}}{\text{成
        功?}}}{\text{成功?}}$};
        & \node (p4)[process, text
        width=90mm]{$\overset{\overset{\text{Call the "pdf\_to\_eps"
        script in the "util" directory to convert to EPS}}{\text{調用“util”目錄中之“pdf\_to\_eps”腳本轉換成
        EPS}}}{\text{调用“util”目录中之“pdf\_to\_eps” 脚本转换成EPS}}$};
        & \\

        % row 7
        &
        &
        & \node (p5)[process, text
        width=40mm]{$\overset{\overset{\text{Delete the single-page PDFs}}{\text{刪除單頁PDF}}}{\text{删除单页PDF}}$};
        & \\

        % row 8
        &
        &
        & \node (end)[startstop]{$\overset{\overset{\text{End}}{\text{結束}}}{\text{结束}}$};
        & \\
    };

    % 拐弯用/拐彎用
    \coordinate[right of=dec2, xshift=-50mm] (dummy1);

    % lines and arrows
    \draw[arrow](start) -- (p1);
    \draw[arrow](p1) -- (dec1);
    \draw[arrow, color=colorYes](dec1)node[anchor=south east, xshift=25mm]{$\overset{\overset{\text{Yes}}{\text{是}}}{\text{是}}$} -- (p2);
    \draw[arrow, color=colorNo](dec1)node[anchor=south east, yshift=-30mm]{$\overset{\overset{\text{No}}{\text{否}}}{\text{否}}$} |- (end);
    \draw[arrow](p2) -- (dec2);
    \draw[arrow, color=colorYes](dec2)node[anchor=north west,
    xshift=2mm,
    yshift=-13mm]{$\overset{\overset{\text{Yes}}{\text{是}}}{\text{是}}$} -- (p3);
    \draw[arrow, color=colorNo](dec2)node[anchor=east,
    xshift=-20mm, yshift=7mm]{$\overset{\overset{\text{No}}{\text{否}}}{\text{否}}$}
    -- (dummy1) |- (end);
    \draw[arrow](p3) -- (dec3);
    \draw[arrow, color=colorYes](dec3)node[anchor=south east,
    xshift=25mm]{$\overset{\overset{\text{Yes}}{\text{是}}}{\text{是}}$}
    -- (p4);
    \draw[arrow, color=colorNo](dec3)node[anchor=south east, yshift=-30mm]{$\overset{\overset{\text{No}}{\text{否}}}{\text{否}}$} |- (end);
    \draw[arrow](p4) -- (p5);
    \draw[arrow](p5) -- (end);

    % % dashed line
    \draw[dashed] (p1) -- (c1);

\end{tikzpicture}
